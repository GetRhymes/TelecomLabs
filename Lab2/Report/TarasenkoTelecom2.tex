\documentclass[10pt,a4paper,oneside]{article}
\usepackage{cmap}
\usepackage[T2A]{fontenc}
\usepackage{float}
\usepackage{listings}
\usepackage{csquotes}
\usepackage[utf8]{inputenc}
\usepackage{amsmath}
\usepackage{amsfonts}
\usepackage{amssymb}
\usepackage[english, russian]{babel}%Подключаем русский язык.
\usepackage{graphicx}
\usepackage{geometry} % Меняем поля страницы.
\geometry{left=3cm} %Левое поле.
\geometry{right=2cm} %Правое поле.
\geometry{top=3cm} %Верхнее поле.
\geometry{bottom=2cm} %Нижнее поле.


%Начало документа
\begin{document}

%Создаём титульник.
\begin{titlepage}
\newpage
	%Название ВУЗа и институт.
	\begin{center}
		\Large Санкт-Петербургский Государственный Политехнический Университет\\
		Институт Компьютерных Наук и Технологий\\
	\end{center}
	%Кафедра.
	\begin{center}
		\large\textbf {Высшая школа интеллектуальных систем и суперкомпьютерных технологий}
	\end{center}
	
	%Пропуск места. 
	\vspace{5em}
	%!!!!!!!!!!!!!!!!!!!!!!!!!!!!!!!!!Название работы.
	\begin{center}
		\large{Отчёт по лабораторной работе №2 \\ на тему \\
		\textbf{Гармоники} }
	\end{center}
	
	%Делаем пропуск и пишем студента и преподавателя.
	\vspace{25em}
	\begin{flushright}
		\textbf{Работу выполнил\\}Студент группы 3530901/80203 \\ Тарасенко Н.С.\\
		\textbf{Преподаватель\\}Богач Н.В. 
	\end{flushright}
	
	\vspace{\fill}%В самом низу
	\begin{center}
	Санкт-Петербург, 2021 год	
	\end{center}
\end{titlepage} %Закончили титульный лист.

\section{Настройка проекта}
Перед тем как выполнять задания необходимо настроить проект и сделать все необходимые импорты:

\begin{figure}[H]
        \centering
        \includegraphics[width=0.75\textwidth]{0.png}
        \caption{2}
        \label{fig:first}
\end{figure}

\section{Упражнение номер №1}
1) Необходимо создать класс SawtoothSignal, расширяющий signal и предоставляющий evaluate для оценки пилообразного сигнала.
2) Необходимо вычислить спектр пилообразного сигнала и посмотреть как соотносится его гармоническая структура с треугольным и прямоугольным сигналами.

Пункт 1.1:
Реализуем класс SawtoothSignal. Он расширяет Signal и предоставляет возможность сделать оценку пилообразного сигнала:

\begin{figure}[H]
        \centering
        \includegraphics[width=0.75\textwidth]{1.png}
        \caption{2}
        \label{fig:first}
\end{figure}

Пункт 1.2:
Используя созданный класс создадим пилообразный сигнал: 

\begin{figure}[H]
        \centering
        \includegraphics[width=0.75\textwidth]{2.png}
        \caption{2}
        \label{fig:first}
\end{figure}

Распечатаем спектр: 

\begin{figure}[H]
        \centering
        \includegraphics[width=0.75\textwidth]{3.png}
        \caption{2}
        \label{fig:first}
\end{figure}

Сделаем наложение пилообразного сигнала и прямоугольного:

\begin{figure}[H]
        \centering
        \includegraphics[width=0.75\textwidth]{4.png}
        \caption{2}
        \label{fig:first}
\end{figure}

Из графика видно, что спад пилообразных происходит аналогично спаду прямоугольных. Так же стоит отметить, что пилообразный сигнал включает как четные, так и нечетные гармоники

Сделаем наложение пилообразного сигнала и треугольного: 

\begin{figure}[H]
        \centering
        \includegraphics[width=0.75\textwidth]{5.png}
        \caption{2}
        \label{fig:first}
\end{figure}

В отличии от пилообразных гармоник, у треугольных спад протекает значительно быстрее.
\section{Упражнение номер №2}
Необходимо создать прямоугольный сигнал 1100Гц и вычислить wave с выборками 10 000 кадров в секунду.  Построить спектр и убедиться, что большинство гармоник "завернуты" из-за биений. Проверить слышны ли последствия этого при проигрывании.

Создадим прямоугольный сигнал 1100Гц: 

\begin{figure}[H]
        \centering
        \includegraphics[width=0.75\textwidth]{6.png}
        \caption{2}
        \label{fig:first}
\end{figure}

Как видно из графика, первая гармоника находится в нужном месте. А вот вторая гармоника (5500Hz) совмещена с гармоникой 4500Hz. Следующая гармоника совмещена с гармоникой на 2300Hz.

\section{Упражнение номер №3}
Необходимо взять объект spectrum и распечатать несколько первых значений spectrum.fs . Убедиться что они начинаются с нуля. Затем нужно провести эксперименты:
1) Создать треугольный сигнал с частотой 440Гц и wave длительностью 0.01 секунд. Распечатать сигнал.
2) Создать объект spectrum и распечатать spectrum.hs[0]. Посмотреть каковы амплитуда и фазы этого компонента.
3) Установить spectrum.hs[0] = 100. Проверить как эта операция влияет на сигнал. 

Создали треугольный сигнал и распечатали wave:

\begin{figure}[H]
        \centering
        \includegraphics[width=0.75\textwidth]{7.png}
        \caption{2}
        \label{fig:first}
\end{figure}

Каждый элмент массива hs объекта Spectrum представялет собой комплексное число и соответствует частотоной компоненте: размах пропорционален амплитуде соответствующей компоненты, а угол - это фаза. Как видно из результатов выполнения кода, первый элемент массива hs - комплексное число, близкое к нулю, мнимая часть равна нулю. Присвоим первый элемент 100 и посмотрим, что из этого выйдет.

\begin{figure}[H]
        \centering
        \includegraphics[width=0.75\textwidth]{8.png}
        \caption{2}
        \label{fig:first}
\end{figure}

Как мы можем заметить это привело к вертикальному смещению волны

\section{Упражнение номер №4}
Необходимо реализовать функцию, которая в качестве аргумента принимает spectrum и изменяет его делением каждого элемента hs на соответствующую частоту из fs. Проверить эту функцию на прямоугольном, треугольном и пилообразном сигналах:
1) Вычислить spectrum и распечатать его
2) Изменить spectrum и распечатать его
3) Использовать spectrum.make_wave, чтобы сделать wave из измененного spectrum, прослушать его. Посмотреть как изменился сигнал.

Реализуем метод: 

\begin{figure}[H]
        \centering
        \includegraphics[width=0.75\textwidth]{9.png}
        \caption{2}
        \label{fig:first}
\end{figure}

Создадим треугольный сигнал:

\begin{figure}[H]
        \centering
        \includegraphics[width=0.75\textwidth]{10.png}
        \caption{2}
        \label{fig:first}
\end{figure}

Применим написанную нами функцию к созданному сигналу: 

\begin{figure}[H]
        \centering
        \includegraphics[width=0.75\textwidth]{11.png}
        \caption{2}
        \label{fig:first}
\end{figure}

Распечатаем полученный спектр: 

\begin{figure}[H]
        \centering
        \includegraphics[width=0.75\textwidth]{12.png}
        \caption{2}
        \label{fig:first}
\end{figure}

Фильтр подавляет гармоники, поэтому он действует как фильтр нижних частот.
\section{Упражнение номер №5}
Проверить можно ли найти сигнал, состоящий из четных и нечетных гармоник, спадающих пропорционально 1/(f^2).

Нужно создать сигнал, состоящий из четных и нечетных гармоник, при этом, что эти гармоники падали пропорционально 1/ (f^2) Для этого воспользуемся одним из способов: создадим пилообразный сигнал и выведем его спектр:  

\begin{figure}[H]
        \centering
        \includegraphics[width=0.75\textwidth]{13.png}
        \caption{2}
        \label{fig:first}
\end{figure}

Используем функцию из предыдущего пункта и преобразуем наш сигнал, а после сделаем наложение спектров: 

При наложении видно, что спад происходит как в условии

Отобразим полученную нами wave:

\begin{figure}[H]
        \centering
        \includegraphics[width=0.75\textwidth]{14.png}
        \caption{2}
        \label{fig:first}
\end{figure}

Как видно из графиков, полученных после обработки сигнала, спектр спадает пропорционально как сказано в условии и при этом имеет четные и нечетные графики. Как видно из последнего графика, сигнал перестал быть пилообразным, и стал похожим на синусоидный.

\end{document}